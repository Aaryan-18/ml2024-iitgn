\documentclass{beamer}
\usepackage{tcolorbox}
\usepackage{forloop}
\usepackage{bm}

\usepackage{mathtools}
\mathtoolsset{showonlyrefs}  

%\beamerdefaultoverlayspecification{<+->}
\newcommand{\data}{\mathcal{D}}

\DeclareMathOperator*{\argmin}{arg\,min}

\newcommand\Item[1][]{%
	\ifx\relax#1\relax  \item \else \item[#1] \fi
	\abovedisplayskip=0pt\abovedisplayshortskip=0pt~\vspace*{-\baselineskip}}

\usetheme{metropolis}           % Use metropolis theme


\title{Multivariate Normal Distribution II}
\date{\today}
\author{Nipun Batra}
\institute{IIT Gandhinagar}
\begin{document}
	\maketitle
	
	\urldef\nodeSix\url{http://fourier.eng.hmc.edu/e161/lectures/gaussianprocess/node6.html}
	\urldef\nodeSeven\url{http://fourier.eng.hmc.edu/e161/lectures/gaussianprocess/node7.html}
	
	
	\begin{frame}{Detour: Inverse of partioned symmetric matrix \footnote{Courtesy: \nodeSix}}
		
		Consider an $n\times n$ symmetric matrix $A$ and divide it into four blocks
		
		$
		A = \begin{bmatrix}
		A_{11} & A_{12}\\
		A_{21} & A_{22} \\
		\end{bmatrix} =  \begin{bmatrix}
		A_{11} & A_{12}\\
		A_{12}^T & A_{22} \\
		\end{bmatrix}
		$
		
		For example, let $n=3$, we have
		
		$$
		A =  \begin{bmatrix}
		1 & 2 & 3\\
		2 & 5 & 6 \\
		3 & 6 & 8 \\
		
	\end{bmatrix}
	$$
	
	We could for example have 
	
	$A_{11} = \begin{bmatrix}
	1 & 2 \\
	2 & 5 \\
	\end{bmatrix}$ and $A_{12} = \begin{bmatrix}
	3 \\ 6
	\end{bmatrix}
	$ and $A_{22} = \begin{bmatrix}
	8
	\end{bmatrix}$
\end{frame}

\begin{frame}{Detour: Inverse of partioned symmetric matrix}
	Question: Write $B = A^{-1}$ in terms of the four blocks
	$
	B = \begin{bmatrix}
	B_{11} & B_{12}\\
	B_{12}^T & A_{22} \\
	\end{bmatrix}
	= A^{-1}$
	
	$A_{11}$ and $B_{11} \in R^{p\times p}$ 
	
	$A_{22}$ and $B_{22} \in R^{q\times q}$
	
	$A_{12} = A_{21}^T$ and $B_{12} = B_{21}^T \in R^{p\times q}$
	
	and, p + q = n 
	
\end{frame}

\begin{frame}{Detour: Inverse of partioned symmetric matrix}
	$I_n = AA^{-1} = AB$
	
	$=\begin{bmatrix}
	A_{11} & A_{12}\\
	A_{12}^T & A_{22} \\
	\end{bmatrix} \begin{bmatrix}
	B_{11} & B_{12}\\
	B_{12}^T & B_{22} \\
	\end{bmatrix}
	= \begin{bmatrix}
	A_{11}B_{11} +A_{12}B_{12}^T & A_{11}B_{12} + A_{12}A_{22}\\
	A_{12}^TB_{11} + A_{22}B_{12}^T & A_{12}^TB_{12} + A_{22}B_{22}  \\
	\end{bmatrix} = \begin{bmatrix}
	I_p & 0 \\
	0 & I_q
	\end{bmatrix}$
	
	Thus, we have
	\begin{gather}
	A_{11}B_{11} +A_{12}B_{12}^T = I_p\\
	A_{11}B_{12} + A_{12}A_{22} = 0^{p\times q}\\
	A_{12}^TB_{11} + A_{22}B_{12}^T = 0^{q\times p}\\
	A_{12}^TB_{12} + A_{22}B_{22} = I_q
	\end{gather}
\end{frame}

\begin{frame}{Detour: Inverse of partioned symmetric matrix}
	Moving the expressions around we get the following results.
	
	\begin{gather}
	B_{11} = (A_{11} - A_{12}A_{22}^{-1}A_{12}^T)^{-1} = A_{11}^{-1} + A_{11}^{-1}A_{12}(A_{22} - A_{12}^TA_{11}^{-1}A_{12})^{-1}A_{12}^TA_{11}^{-1}\\
	B_{22} = (A_{22} - A_{12}^TA_{11}^{-1}A_{12})^{-1} = A_{22}^{-1} + A_{22}^{-1}A_{12}^T(A_{11} - A_{12}A_{22}^{-1}A_{12}^T)^{-1}A_{12}A_{22}^{-1}\\
	B_{12}^T = - A_{22}^{-1}A_{12}^T ( A_{11} - A_{12} A_{22}^{-1} A_{12}^T)^{-1}\\\\
	B_{12}^T = - A_{11}^{-1}A_{12}^T ( A_{22} - A_{12} ^TA_{11}^{-1} A_{12})^{-1}
	\end{gather}
\end{frame}

\begin{frame}{Determinant of Partitioned Symmetric Matrix}
	
	\textbf{Theorem:} Determinant of a partitioned symmetric matrix can be written as follows
	\begin{gather}
	\vert A \vert 
	= \bigg| 
	\begin{bmatrix}
	A_{11}&A_{12}\\
	A_{21}&A_{22}
	\end{bmatrix}
	\bigg| \\
	=\vert A_{11}\vert \vert A_{22}-A_{12}^TA_{11}^{-1}A_{12}\vert \\
	=\vert A_{22}\vert \vert
	A_{11}-A_{12}A_{22}^{-1}A_{12}^T \vert 
	\end{gather}
\end{frame}

\begin{frame}{Determinant of Partitioned Symmetric Matrix}
	\textbf{Proof:} Note that
	\begin{gather}
	A = 
	\begin{bmatrix}
	A_{11}&A_{12} \\
	A_{21}&A_{22}
	\end{bmatrix}
	= 
	\begin{bmatrix}
	A_{11}& 0 \\
	A_{12}^T& I
	\end{bmatrix}
	\begin{bmatrix}
	I & A_{11}^{-1}A_{12} \\
	0 & A_{22} - A_{12}^TA_{11}^{-1}A_{12}
	\end{bmatrix}\\
	= 
	\begin{bmatrix}
	I & A_{12} \\
	0 & A_{22}
	\end{bmatrix}
	\begin{bmatrix}
	A_{11} - A_{12}A_{22}^{-1}A_{12}^T & 0\\
	A_{22}^{-1}A_{21}  & I
	\end{bmatrix}
	\end{gather} 
	
	The theorem is proved as we also know that
	\begin{equation}
	\vert AB\vert=\vert A\vert  \vert B\vert 
	\end{equation}
	
	and
	\begin{equation}
	\left\vert 
	\begin{matrix}
	B&0 \\
	C&D
	\end{matrix} 
	\right\vert
	=
	\left\vert 
	\begin{matrix}
	B&C \\
	0&D
	\end{matrix} 
	\right\vert
	=
	\vert B\vert\
	\vert D\vert 
	\end{equation}	
\end{frame}

\begin{frame}{Marginalisation and Conditional of multivariate normal\footnote{Courtesy: \nodeSeven.}}
	Assume an n-dimensional random vector
	
	\begin{equation}
	{\bf x}=\begin{bmatrix}{\bf x}_1 & {\bf x}_2\end{bmatrix} 
	\end{equation}
	
	has a normal distribution $N({\bf x},\mu,\Sigma)$ with
	\begin{gather}\mu=
	\begin{bmatrix}
	\mu_1 \\
	\mu_2
	\end{bmatrix} 
	\text{and }
	\Sigma = \begin{bmatrix}
	\Sigma_{11}& \Sigma_{12}\\
	\Sigma_{21}&\Sigma_{22}
	\end{bmatrix} 
	\end{gather}
	
	where ${\bf x}_1$ and ${\bf x}_2$ are two subvectors of respective dimensions $p$ and $q$ with $p+q=n$. Note that $\Sigma=\Sigma^T$, and $\Sigma_{21}=\Sigma_{21}^T$.
\end{frame}

\begin{frame}{Marginalisation and Conditional of multivariate normal}
	\textbf{Theorem:}
	
	\textbf{part a:} The marginal distributions of ${\bf x}_1$ and ${\bf x}_2$ are also normal with mean vector $\mu_i$ and covariance matrix $\Sigma_{ii}$ ($i=1,2$), respectively.
	
	\textbf{part b:} The conditional distribution of ${\bf x}_i$ given ${\bf x}_j$ is also normal with mean vector
	
\end{frame}

\begin{frame}{Marginalisation and Conditional of multivariate normal}
	\textbf{Proof:}
	
	The joint density of ${\bf x}$ is:
	
	\begin{equation}
	f({\bf x})=f({\bf x}_1,{\bf x}_2)=\frac{1}{(2\pi)^{n/2\vert\Sigma\vert^{1/2}}}exp[-\frac{1}{2}Q({\bf x}_1,{\bf x}_2)] 
	\end{equation}
	
	where $Q$ is defined as
	\begin{gather}
	Q({\bf x}_1,{\bf x}_2) = ({\bf x}-\mu)^T\Sigma^{-1}({\bf x}-\mu)\\
	= [({\bf x}_1-\mu_1)^T, ({\bf x}_2-\mu_2)^T] 
	\begin{bmatrix}
	\Sigma^{11} & \Sigma^{12}\\
	\Sigma^{21} & \Sigma^{22}
	\end{bmatrix}
	\begin{bmatrix}
	{\bf x}_1-\mu_1 \\
	{\bf x}_2-\mu_2
	\end{bmatrix}\\
	= ({\bf x}_1-\mu_1)^T\Sigma^{11}({\bf x}_1-\mu_1)+ 2({\bf x}_1-\mu_1)^T\Sigma^{12}({\bf x}_2-\mu_2) + ({\bf x}_2-\mu_2)^T \cdots \\
	\cdots \Sigma^{22}({\bf x}_2-\mu_2)
	\end{gather}
\end{frame}

\begin{frame}{Marginalisation and Conditional of multivariate normal}
	Here we have assumed 
	$$	
	\Sigma^{-1}=\left[\begin{array}{cc}
	{\Sigma_{11}} & {\Sigma_{12}} \\
	{\Sigma_{21}} & {\Sigma_{22}}
	\end{array}\right]^{-1}=\left[\begin{array}{cc}
	{\Sigma^{11}} & {\Sigma^{12}} \\
	{\Sigma^{21}} & {\Sigma^{22}}
	\end{array}\right]
	$$
	
	According to inverse of a partitioned symmetric matrix we have, 
	\begin{gather}
	\Sigma^{11}=\left(\Sigma_{11}-\Sigma_{12} \Sigma_{22}^{-1} \Sigma_{12}^{T}\right)^{-1}=\Sigma_{11}^{-1}+\Sigma_{11}^{-1} \Sigma_{12}\left(\Sigma_{22}-A_{12}^{T} \Sigma_{11}^{T} \Sigma_{12}\right)^{-1} \Sigma_{12}^{T} \Sigma_{11}^{-1}\\
	\Sigma^{22}=\left(\Sigma_{22}-\Sigma_{12}^{T} \Sigma_{11}^{-1} \Sigma_{12}\right)^{-1}=\Sigma_{22}^{-1}+\Sigma_{22}^{-1} \Sigma_{12}^{T}\left(\Sigma_{11}-\Sigma_{12} \Sigma_{22}^{-1} \Sigma_{12}^{T}\right)^{-1} \Sigma_{12} \Sigma_{22}^{-1}\\
	\Sigma^{12}=-\Sigma_{11}^{-1} \Sigma_{12}\left(\Sigma_{22}-\Sigma_{12}^{T} \Sigma_{11}^{-1} \Sigma_{12}\right)^{-1}=\left(\Sigma^{21}\right)^{T}
	\end{gather}
	
\end{frame}

\begin{frame}{Marginalisation and Conditional of multivariate normal}
	Substituting the second expression for $\Sigma^{11}$, first expression for $\Sigma^{22}$, and $\Sigma^{12}$ into $Q({\bf x}_1,{\bf x}_2)$ to get:
	
	\begin{gather}
	\begin{aligned}
	Q\left(\mathbf{x}_{1}, \mathbf{x}_{2}\right)=&\left(\mathbf{x}_{1}-\mu_{1}\right)^{T}\left[\Sigma_{11}^{-1}+\Sigma_{11}^{-1} \Sigma_{12}\left(\Sigma_{22}-A_{12}^{T} \Sigma_{11}^{-1} \Sigma_{12}\right)^{-1} \Sigma_{12}^{T} \Sigma_{11}^{-1}\right]\left(\mathbf{x}_{1}-\mu_{1}\right) \\
	&-2\left(\mathbf{x}_{1}-\mu_{1}\right)^{T}\left[\Sigma_{11}^{-1} \Sigma_{12}\left(\Sigma_{22}-\Sigma_{12}^{T} \Sigma_{11}^{-1} \Sigma_{12}\right)^{-1}\right]\left(\mathbf{x}_{2}-\mu_{2}\right) \\
	&+\left(\mathbf{x}_{2}-\mu_{2}\right)^{T}\left[\left(\Sigma_{22}-\Sigma_{12}^{T} \Sigma_{11}^{-1} \Sigma_{12}\right)^{-1}\right]\left(\mathbf{x}_{2}-\mu_{2}\right)\\
	=&\left(\mathbf{x}_{1}-\mu_{1}\right)^{T} \Sigma_{11}^{-1}\left(\mathbf{x}_{1}-\mu_{1}\right) \\
	&\left.+\left(\mathbf{x}_{1}-\mu_{1}\right)^{T} \Sigma_{11}^{-1} \Sigma_{12}\left(\Sigma_{22}-A_{12}^{T} \Sigma_{11}^{-1} \Sigma_{12}\right)^{-1} \Sigma_{12}^{T} \Sigma_{11}^{-1}\right]\left(\mathbf{x}_{1}-\mu_{1}\right) \\
	&-2\left(\mathbf{x}_{1}-\mu_{1}\right)^{T}\left[\Sigma_{11}^{-1} \Sigma_{12}\left(\Sigma_{22}-\Sigma_{12}^{T} \Sigma_{11}^{-1} \Sigma_{12}\right)^{-1}\right]\left(\mathbf{x}_{2}-\mu_{2}\right) \\
	&+\left(\mathbf{x}_{2}-\mu_{2}\right)^{T}\left[\left(\Sigma_{22}-\Sigma_{12}^{T} \Sigma_{11}^{-1} \Sigma_{12}\right)^{-1}\right]\left(\mathbf{x}_{2}-\mu_{2}\right)
	\end{aligned}\\
	\end{gather})
\end{frame}

\begin{frame}{Marginalisation and Conditional of multivariate normal}
	\begin{gather}
	\begin{aligned}
	=&\left(\mathbf{x}_{1}-\mu_{1}\right)^{T} \Sigma_{11}^{-1}\\
	&+\left[\left(\mathbf{x}_{2}-\mu_{2}\right)-\Sigma_{12}^{T} \Sigma_{11}^{-1}\left(\mathbf{x}_{1}-\mu_{1}\right)\right]^{T}\left(\Sigma_{22}-\Sigma_{12}^{T} \Sigma_{11}^{-1} \Sigma_{12}\right)^{-1}\left[\left(\mathbf{x}_{2}-\mu_{2}\right)-\Sigma_{12}^{T} \Sigma_{11}^{-1}\left(\mathbf{x}_{1}-\mu_{1}\right)\right]
	\end{aligned}
	\end{gather}
	The last equal sign is due to the following equations for any vectors $u$ and $v$ and a symmetric matrix $A=A^T$:
	
	\begin{gather}
	\begin{aligned}
	& u^{T} A u-2 u^{T} A v+v^{T} A v=u^{T} A u-u^{T} A v-u^{T} A v+v^{T} A v \\
	=& u^{T} A(u-v)-(u-v)^{T} A v=u^{T} A(u-v)-v^{T} A(u-v) \\
	=&(u-v)^{T} A(u-v)=(v-u)^{T} A(v-u)
	\end{aligned} 
	\end{gather}
\end{frame}

\begin{frame}{Marginalisation and Conditional of multivariate normal}
	We define  
	$b \triangleq \mu_{2}+\Sigma_{12}^{T} \Sigma_{11}^{-1}\left(\mathbf{x}_{1}-\mu_{1}\right)$
	\[
	A \triangleq \Sigma_{22}-\Sigma_{12}^{T} \Sigma_{11}^{-1} \Sigma_{12}
	\]
	
	and 
	$$\left\{\begin{array}{l}{Q_{1}\left(\mathrm{x}_{1}\right) \quad \triangleq\left(\mathrm{x}_{1}-\mu_{1}\right)^{T} \Sigma_{1}^{-1}\left(\mathrm{x}_{1}-\mu_{1}\right)} \\ {Q_{2}\left(\mathrm{x}_{1}, \mathrm{x}_{2}\right) \triangleq\left[\left(\mathrm{x}_{2}-\mu_{2}\right)-\Sigma_{12}^{T} \Sigma_{11}^{-1}\left(\mathrm{x}_{1}-\mu_{1}\right)\right]^{T}\left(\Sigma_{22}-\Sigma_{12}^{T} \Sigma_{11}^{-1} \Sigma_{12}\right)^{-1}\left[\left(\mathrm{x}_{2}-\mu_{2}\right)-\Sigma_{12}^{T} \Sigma_{11}^{-1}\left(\mathrm{x}_{1}-\mu_{1}\right)\right]} \\ {=\left(\mathrm{x}_{2}-b\right)^{T} A^{-1}\left(\mathrm{x}_{2}-b\right)}\end{array}\right.$$
	
	and get 
	$$Q\left(\mathbf{x}_{1}, \mathbf{x}_{2}\right)=Q_{1}\left(\mathbf{x}_{1}\right)+Q_{2}\left(\mathbf{x}_{1}, \mathbf{x}_{2}\right)$$
\end{frame}


\begin{frame}{Marginalisation and Conditional of multivariate normal}
	Now the joint distribution can be written as: 
	
	$$\begin{aligned} f(\mathrm{x}) &=f\left(\mathrm{x}_{1}, \mathrm{x}_{2}\right)=\frac{1}{(2 \pi)^{n / 2}\vert\Sigma\vert^{1 / 2}} \exp \left[-\frac{1}{2} Q\left(\mathrm{x}_{1}, \mathrm{x}_{2}\right)\right] \\ &=\frac{1}{(2 \pi)^{n / 2}\left|\Sigma_{11}\right|^{1 / 2}\left|\Sigma_{22}-\Sigma_{12}^{T} \Sigma_{11}^{-1} \Sigma_{12}\right|^{1 / 2}} \exp \left[-\frac{1}{2} Q\left(\mathrm{x}_{1}, \mathrm{x}_{2}\right)\right] \\ &=\frac{1}{(2 \pi)^{p / 2}\left|\Sigma_{11}\right|^{1 / 2}} \exp \left[-\frac{1}{2}\left(\mathrm{x}_{1}-\mu_{1}\right)^{T} \Sigma_{11}^{-1}\left(\mathrm{x}_{1}-\mu_{1}\right)\right] \frac{1}{(2 \pi)^{q / 2}|A|^{1 / 2}} \exp \left[-\frac{1}{2}\left(\mathrm{x}_{2}-b\right)^{T} A^{-1}\left(\mathrm{x}_{2}-b\right)\right] \\ &=N\left(\mathrm{x}_{1}, \mu_{1}, \Sigma_{11}\right) N\left(\mathrm{x}_{2}, b, A\right) \end{aligned}$$
	
	The third equal sign is due to Determinant of a partitioned symmetric matrix:
	
	$|\Sigma|=\left|\Sigma_{11}\right|\left|\Sigma_{22}-\Sigma_{12}^{T} \Sigma_{11}^{-1} \Sigma_{12}\right|$
\end{frame}

\begin{frame}{Marginalisation and Conditional of multivariate normal}
	The marginal distribution of ${\bf x}_1$ is 
	$$f_{1}\left(\mathbf{x}_{1}\right)=\int f\left(\mathbf{x}_{1}, \mathbf{x}_{2}\right) d \mathbf{x}_{2}=\frac{1}{(2 \pi)^{p / 2}\left|\Sigma_{11}\right|^{1 / 2}} \exp \left[-\frac{1}{2}\left(\mathbf{x}_{1}-\mu_{1}\right)^{T} \Sigma_{11}^{-1}\left(\mathbf{x}_{1}-\mu_{1}\right)\right]$$
	
	and the conditional distribution of ${\bf x}_2$ given ${\bf x}_1$ is 
	
	$$f_{2 | 1}\left(\mathrm{x}_{2} | \mathrm{x}_{1}\right)=\frac{f\left(\mathrm{x}_{1}, \mathrm{x}_{2}\right)}{f\left(\mathrm{x}_{1}\right)}=\frac{1}{(2 \pi)^{q / 2}|A|^{1 / 2}} \exp \left[-\frac{1}{2}\left(\mathrm{x}_{2}-b\right)^{T} A^{-1}\left(\mathrm{x}_{2}-b\right)\right]$$
\end{frame}

\begin{frame}{Marginalisation and Conditional of multivariate normal}
	with
	
	$$b=\mu_{2}+\Sigma_{12}^{T} \Sigma_{11}^{-1}\left(\mathbf{x}_{1}-\mu_{1}\right)$$
	\[
	A=\Sigma_{22}-\Sigma_{12}^{T} \Sigma_{11}^{-1} \Sigma_{12}
	\]
\end{frame}


%\begin{frame}{Marginalisation of bivariate normal}
%	Complete derivation from \textcolor{red}{Where?}
%\end{frame}
%
%\begin{frame}{Conditional Normal Distribution}
%
%\end{frame}
\end{document}