\documentclass{beamer}
\usepackage{tcolorbox}

\usepackage{tikz}
\usetikzlibrary{arrows,backgrounds}
\usepgflibrary{shapes.multipart}

%\beamerdefaultoverlayspecification{<+->}
\newcommand{\data}{\mathcal{D}}

\DeclareMathOperator*{\argmin}{arg\,min}

\newcommand\Item[1][]{%
	\ifx\relax#1\relax  \item \else \item[#1] \fi
	\abovedisplayskip=0pt\abovedisplayshortskip=0pt~\vspace*{-\baselineskip}}


\usetheme{metropolis}           % Use metropolis theme


\title{Bayesian Linear Regression}
\date{\today}
\author{Nipun Batra}
\institute{IIT Gandhinagar}
\begin{document}
  \maketitle
  
  
  
\section{Perceptron}

\begin{frame}{A simple perceptron}
\begin{figure}[h]
	\begin{center}
		\begin{tikzpicture}
		\tikzstyle{place}=[circle, draw=black, minimum size = 8mm]
		
		% Input
		\foreach \x in {1,...,2}
		\draw node at (0, -\x*1.25) [place] (first_\x) {$x_\x$};
		\draw node at (0, -3*1.25) [place] (bias) {$1$};
		
			
		% Output
		\draw node at (3, -1.25*2) [place] (out) {$Out$};
		
		% Connections
		\draw [->] (first_1) to (out);
		\draw [->] (first_2) to (out);
		\draw [->] (bias) to (out);
		
		\node at (1, -1.25) [black, ] {$w_1$};
		\node at (1, -2.25) [black, ] {$w_2$};
		\node at (1, -3.7) [black, ] {$b$};

		% Text
		\node at (0, -5) [black, ] {Input Layer};
		\node at (3, -5) [black, ] {Output Layer};
		\end{tikzpicture}
		\caption{Illustration of a perceptron}
		\label{fig:illustration_multilayer_perceptron}
	\end{center}
\end{figure}
\end{frame}

\end{document}