
\documentclass{beamer}
\usepackage{tcolorbox}
\usepackage{caption}
\usepackage{pgfplots}
\usepackage{grffile}
%\beamerdefaultoverlayspecification{<+->}
\newcommand{\data}{\mathcal{D}}

\DeclareMathOperator*{\argmin}{arg\,min}

\newcommand\Item[1][]{%
	\ifx\relax#1\relax  \item \else \item[#1] \fi
	\abovedisplayskip=0pt\abovedisplayshortskip=0pt~\vspace*{-\baselineskip}}


\usetheme{metropolis}           % Use metropolis theme


\title{Linear Rergression Time Complexity Calculation}
\date{\today}
\author{Nipun Batra}
\institute{IIT Gandhinagar}
\begin{document}
  \maketitle
  
  
  
  
% \section{Linear Regression}

\begin{frame}{Normal Equation}

	\begin{itemize}[<+->]
		\item Consider $X\in \mathcal{R}^{N\times D}$
		\item $N$ examples and $D$ dimensions
		\item What is the time complexity of solving the normal equation $\hat{\theta} = (X^TX)^{-1}X^Ty$?
	\end{itemize}
	
	

\end{frame}

\begin{frame}{Normal Equation}

\begin{itemize}[<+->]
	\item $X$ has dimensions $N\times D$, $X^T$ has dimensions $D \times N$
	\item $X^TX$ is a matrix product of matrices of size: $D \times N$ and $N \times D$, which is $\mathcal{O}(D^{2}N)$
	\item Inversion of $X^TX$ is an inversion of a $D\times D$ matrix, which is $\mathcal{O}(D^{3})$
	\item $X^Ty$ is a matrix vector product of size $D \times N$ and $N \times 1$, which is $\mathcal{O}(DN)$
	\item $(X^TX)^{-1}X^Ty$ is a matrix product of a  $D\times D$ matrix and $D \times 1$ matrix, which is $\mathcal{O}(D^2)$
	\item Overall complexity: $\mathcal{O}(D^{2}N)$ + $\mathcal{O}(D^{3})$ + $\mathcal{O}(DN)$ + $\mathcal{O}(D^2)$ = $\mathcal{O}(D^{2}N)$ + $\mathcal{O}(D^{3})$
	\item Scales cubic in the number of columns/features of $X$
\end{itemize}



\end{frame}


\begin{frame}{Gradient Descent}
Start with random values of $\theta_{0}$ and $\theta_{1}$\\
Till convergence
\begin{itemize}[<+->]
	\item $\theta_{0} = \theta_{0} - \cfrac{\partial}{\partial \theta_{0}} (\sum \epsilon_{i}^{2}) $
	\item $\theta_{1} = \theta_{1} - \cfrac{\partial}{\partial \theta_{1}} (\sum \epsilon_{i}^{2}) $
	\item Question: Can you write the above for $D$ dimensional data in vectorised form?
	\item	\(\theta_{0}=\theta_{0}-\alpha \frac{\partial}{\partial \theta_{0}}\left(y-X\theta\right)^{\top}\left(y-X\theta\right)\)
	\(\theta_{1}=\theta_{1}-\alpha \frac{\partial}{\partial \theta_{1}}\left(y-X\theta\right)^{\top}\left(y-X\theta\right)\) 
	\\ $\vdots$
	\\	\(\theta_{D}=\theta_{D}-\alpha \frac{\partial}{\partial \theta_{D}}\left(y-X\theta\right)^{\top}\left(y-X\theta\right)\)
	\item \(\theta=\theta - \alpha \frac{\partial}{\partial \theta}\left(y-X\theta\right)^{\top}\left(y-X\theta\right)\) 

\end{itemize}
\end{frame}

\begin{frame}{Gradient Descent}


\(\frac{\partial}{\partial \theta}(y-X \theta)^{\top}(y-X \theta)\)
\\ \(=\frac{\partial}{\partial \theta}\left(y^{\top}-\theta^{\top} X^{\top}\right)(y-X \theta)\)
\\ \(=\frac{\partial}{\partial \theta}\left(y^{\top} y-\theta^{\top} X^{\top} y-y^{\top} x \theta+\theta^{\top} X^{\top} X \theta\right)\)
\\ \(=-2 X^{\top} y+2 X^{\top} x \theta\)
\\ \(=2 X^{\top}(X \theta-y)\)
	

\end{frame}

\begin{frame}{Gradient Descent}


We can write the vectorised update equation as follows, for each iteration

\(\theta=\theta - \alpha X^{\top}(X \theta-y)\) 


\pause For $t$ iterations, what is the computational complexity of our gradient descent solution?

\pause Hint, rewrite the above as: \(\theta=\theta - \alpha X^{\top}X \theta+ \alpha X^{\top}y\) 

\pause Complexity of computing $X^{\top}y$ is $\mathcal{O}(DN)$

\pause Complexity of computing $\alpha X^{\top}y$ once we have $X^{\top}y$ is $\mathcal{O}(DN)$ since  $X^{\top}y$ has $DN$ entries

\pause Complexity of computing $X^{\top}X$ is $\mathcal{O}(D^2N)$ and then multiplying with $\alpha$ is  $\mathcal{O}(D^2)$

\pause All of the above need only be calculated once!



\end{frame}



\begin{frame}{Gradient Descent}
\pause For each of the $t$ iterations, we now need to first multiply  $\alpha X^{\top}X$ with $\theta$ which is matrix multiplication of a $D \times D$ matrix with a $D \times 1$, which is $\mathcal{O}(D^2)$ 

\pause The remaining subtraction/addition can be done in $\mathcal{O}(D)$ for each iteration.

\pause What is overall computational complexity?

\pause $\mathcal{O}(tD^2)$ + $\mathcal{O}(D^2N) = \mathcal{O}((t+N)D^2)$
\end{frame}

\begin{frame}{Gradient Descent (Alternative)}
\pause If we do not rewrite the expression
\(\theta=\theta - \alpha X^{\top}(X \theta-y)\) 

For each iteration, we have:
\begin{itemize}[<+->]
	\item Computing $X\theta$ is $\mathcal{O}(ND)$
	\item Computing $X\theta - y$ is $\mathcal{O}(N)$
	\item Computing $\alpha X^{\top}$ is $\mathcal{O}(ND)$
	\item Computing $\alpha X^{\top}(X\theta - y)$ is $\mathcal{O}(ND)$
	\item Computing \(\theta=\theta - \alpha X^{\top}(X \theta-y)\) is $\mathcal{O}(N)$
\end{itemize}

\pause What is overall computational complexity?

\pause $\mathcal{O}(NDt)$
\end{frame}

\end{document}