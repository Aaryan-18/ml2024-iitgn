\documentclass{beamer}
\usepackage{tcolorbox}

%\beamerdefaultoverlayspecification{<+->}
\newcommand{\data}{\mathcal{D}}

\DeclareMathOperator*{\argmin}{arg\,min}

\newcommand\Item[1][]{%
	\ifx\relax#1\relax  \item \else \item[#1] \fi
	\abovedisplayskip=0pt\abovedisplayshortskip=0pt~\vspace*{-\baselineskip}}


\usetheme{metropolis}           % Use metropolis theme


\title{Linear Regression}
\date{\today}
\author{Nipun Batra}
\institute{IIT Gandhinagar}
\begin{document}
  \maketitle
  
  
  
% \section{Linear Regression}

\begin{frame}{Maths for ML}

$$
\epsilon = \begin{bmatrix}
    \epsilon_{1}   \\
    \epsilon_{2}   \\
    \dots \\
    \epsilon{N}
\end{bmatrix}    
$$

Given a vector of $\epsilon$, we can calculate $\sum \epsilon_{i}^{2}$ using $\epsilon^{T}\epsilon$
\end{frame}

\begin{frame}{Maths for ML}

$$
(AB)^{T} = B^{T}A^{T}    
$$



For a scalar s
$$
S = S^{T}    
$$

\end{frame}


\begin{frame}{Maths for ML}

    
    
  \begin{equation*}
      \theta = \begin{bmatrix}
  \theta_{1}\\
  \theta_{2}\\
  \vdots\\
  \theta_{N}
    \end{bmatrix}   
  
  \end{equation*} 


   
 
  \begin{equation*}
        \frac{\partial S}{\partial \theta} = \begin{bmatrix}
  \frac{\partial S}{\partial \theta_{1}}\\
  \frac{\partial S}{\partial \theta_{2}}\\
    \vdots\\
  \frac{\partial S}{\partial \theta_{N}}\\
    \end{bmatrix}
  \end{equation*}
      
    
    

\end{frame}


\begin{frame}{Maths for ML}
  $A$ is a matrix.\\
  $\theta$ is a vector.\\
  $A\theta$ is a scalar.\\
  
  \begin{equation*}
      \theta = \begin{bmatrix}
      \theta_{1}\\
      \theta_{2}
      \end{bmatrix}
  \end{equation*}
  
    \begin{equation*}
      A = \begin{bmatrix}
      A_{1}& A_{2}
      \end{bmatrix}
  \end{equation*}
  
    \begin{equation*}
      A\theta = 
      A_{1}\theta_{1}+A_{2}\theta_{2}
      
  \end{equation*}
  
\end{frame}

\begin{frame}{maths for ML}

\begin{equation*}
    \frac{\partial A\theta}{\partial \theta} = \begin{bmatrix}
    \frac{\partial}{\partial \theta_{1} }(A_{1}\theta_{1}+A_{2}\theta_{2}) \\
    \frac{\partial}{\partial \theta_{2} }(A_{1}\theta_{1}+A_{2}\theta_{2}) \\
    \end{bmatrix}
    = \begin{bmatrix}
    A_{1}\\A_{2}
    \end{bmatrix}
     = A^{T}
\end{equation*}
    
\end{frame}


\begin{frame}{maths for ML}


Assume $Z$ is of format $C^{T}C$

\begin{equation*}
    X = \begin{bmatrix}
    a&b\\
    c&d
    \end{bmatrix}
\end{equation*}

\begin{equation*}
    X^{T} = \begin{bmatrix}
    a&c\\
    b&d
    \end{bmatrix}
\end{equation*}

\begin{equation*}
    Z = X^{T}X =  \begin{bmatrix}
    a^{2}+c^{2}&ab+cd\\
    ab+cd&b^{2}+d^{2}
    \end{bmatrix}
\end{equation*}

$Z$ has a property $Z_{ij}=z_{ji}$ \implies $Z^{T}=Z$

\begin{equation*}
    \frac{\partial}{ \partial \theta} (\theta^{T}Z\theta)
\end{equation*}
    
\end{frame}


\begin{frame}{Maths for ML}
    \begin{equation*}
        \theta = \begin{bmatrix}
        \theta_{1}\\
        \theta_{2}
        \end{bmatrix}
    \end{equation*}
    
    \begin{equation*}
        \theta^{T}Z\theta = a\theta_{1}^{2} + 2b\theta_{1}\theta_{2}+c\theta_{2}^{2} 
    \end{equation*}

The term $\theta^{T}Z\theta$ is a scalar.

\end{frame}


\begin{frame}{Maths for ML}

    % \begin{align}
    % \begin{center}
    %     \begin{split}
    %     \frac{ \partial \theta^{T}Z\theta}{\partial \theta} &= \frac{\partial}{\partial \theta} \begin{bmatrix}
    %         a\theta_{1}^{2}+2b\theta_{1}\theta_{2}+c\theta_{2}^{2}
            
    %     \end{bmatrix}\\
        
    %     &=\begin{bmatrix}
    %     \frac{\partial}{ \partial \theta_{1}}a\theta_{1}^{2}+2b\theta_{1}\theta_{2}+c\theta_{2}^{2}\\
    %     \frac{\partial}{ \partial \theta_{1}}a\theta_{1}^{2}+2b\theta_{1}\theta_{2}+c\theta_{2}^{2}
    %     \end{bmatrix}\\
    %     &=
    %     \begin{bmatrix}
    %     2a\theta_{1}+2b\theta_{2}\\
    %     2b\theta_{2}+2c\theta_{2}\\
    %     \end{bmatrix}\\
    %     &=2\begin{bmatrix}
    %     a&b\\
    %     b&c
    %     \end{bmatrix}
    %     \begin{bmatrix}
    %     \theta_{1}\\
    %     \theta_{2}
    %     \end{bmatrix} = =2Z\theta\\
        
    %     \end{split}
    %     \end{center}
    % \end{align}
        
% \begin{align}
% \label{eqn*:eqlabel}
% \begin{split}
%   \frac{ \partial \theta^{T}Z\theta}{\partial \theta} &= \frac{\partial}{\partial \theta} \begin{bmatrix}
%             a\theta_{1}^{2}+2b\theta_{1}\theta_{2}+c\theta_{2}^{2}
            
%         \end{bmatrix} \\
% &=\begin{bmatrix}
%         \frac{\partial}{ \partial \theta_{1}}a\theta_{1}^{2}+2b\theta_{1}\theta_{2}+c\theta_{2}^{2}\\
%         \frac{\partial}{ \partial \theta_{1}}a\theta_{1}^{2}+2b\theta_{1}\theta_{2}+c\theta_{2}^{2}
%         \end{bmatrix}\\
% \epsilon^{T}\epsilon &= (y^{T} - \theta^{T}X^{T})(y - X\theta)\\
% &=y^{T}y - \theta^{T}X^{T}y - y^{T}X\theta+\theta^{T}X^{T}X\theta\\
% &=y^{T}y - 2y^{T}X\theta+\theta^{T}X^{T}X\theta
% \end{split}
% \end{align}

    
\end{frame}


\end{document}
