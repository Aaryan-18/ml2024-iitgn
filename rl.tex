\documentclass{beamer}
\usepackage{notation}
\usepackage{forloop}
\usepackage{hyperref}

\usepackage{mathtools}  
\mathtoolsset{showonlyrefs}  
% \usepackage[dvipsnames]{xcolor}

\usetheme{metropolis}           % Use metropolis theme
\title{Reinforcement Learning}
\date{\today}
\author{Nipun Batra}
\institute{IIT Gandhinagar}
% \hypersetup{colorlinks=true,linkcolor=magenta,filecolor=magenta,urlcolor=magenta,
% 	linkbordercolor=magenta,
% pdfborderstyle={/S/U/W 1}
% }
\begin{document}
	\maketitle
	
	
	
	\section{Problem Setting}
	\begin{frame}{Learning}
		We have three main types of learning algorithms:
		
		\begin{itemize}
			\item Supervised Learning (SL)
			\item Unsupervised Learning (UL)
			\item Reinforcement Learning (RL)
		\end{itemize}
	\end{frame}

\begin{frame}
	\urldef\diff\url{https://classroom.udacity.com/courses/ud600/lessons/4100878601/concepts/6512308540923}
	
	Broadly, we can take three basic examples where each of these method is used. \footnote{https://classroom.udacity.com/courses/ud600/lessons/4100878601/concepts/6512308540923}
	
	\begin{itemize}
		\item (SL) - Image Classification
		\begin{itemize}
			\item Data available - $<y, \ x>$ pairs
			\item Goal - Function Approximation $f$ where $f : x \rightarrow y$
		\end{itemize}
		\item (UL) - Clustering
		\begin{itemize}
			\item Data available - $<x>$
			\item Goal - Clustering - $f$ where $f : x \rightarrow z$ and, $z$ is a compact discription of $x$(s).
		\end{itemize}
		\item (RL) - Play a Game
		\begin{itemize}
			\item Data available - $<r, \ x>$ pairs
			\item Goal - ****
		\end{itemize}
	\end{itemize}
\end{frame}

\begin{frame}
	\urldef\rlOne\url{http://www0.cs.ucl.ac.uk/staff/d.silver/web/Teaching_files/intro_RL.pdf}
	
	Some of the characteristic features of Reinforcement Leaning problems: \footnote{Take from \rlOne.}
	
	\begin{itemize}
		\item There is no supervisor, only a reward signal.
		\item Feedback is delayed, not instantaneous.
		\item Time really matters (sequential, non i.i.d. data)
		\item Agent's actions affect the subsequent data it receives.
	\end{itemize}
\end{frame}

\begin{frame}
	\urldef\rl\url{http://www0.cs.ucl.ac.uk/staff/d.silver/web/Teaching.html}
	
	This might be helpful \rl.
	
\end{frame}
	
	
\end{document}