\documentclass{beamer}
\usepackage{tcolorbox}

%\beamerdefaultoverlayspecification{<+->}
\newcommand{\data}{\mathcal{D}}
\newcommand\Item[1][]{%
	\ifx\relax#1\relax  \item \else \item[#1] \fi
	\abovedisplayskip=0pt\abovedisplayshortskip=0pt~\vspace*{-\baselineskip}}


\usetheme{metropolis}           % Use metropolis theme


\title{Bayesian Linear Regression}
\date{\today}
\author{Nipun Batra}
\institute{IIT Gandhinagar}
\begin{document}
  \maketitle
  
  
  
\section{MLE, MAP, Bayesian}

\begin{frame}{Bayes Rule - 1}
\begin{itemize}
	
	
	\item $P(A|B)P(B) = P(B|A)P(A)$
	\item Let us consider an example from Wikipedia:
	\begin{itemize}
		\item A particular drug is 99\% sensitive and 99\% specific
		\item i.e. test will produce 99\% true positive results for drug users and 99\% true negative results for non-drug users
		\item 0.5\% of people are users of the drug
		\item Question: What is the probability that a randomly selected individual with a positive test is a drug user?
	\end{itemize}
	
\end{itemize}
\end{frame}

\begin{frame}{Bayes Rule - 2}
\begin{itemize}
	
\item Test will produce 99\% true positive results for drug users and 99\% true negative results for non-drug users $\implies$
\begin{itemize}
	\item $P(Test=+|User=Drug) = 0.99$, or, $P(+|User) = 0.99$
	\item and $P(-|\overline{User}) =0.99$ 
\end{itemize}  
		\item 0.5\% of people are users of the drug $\implies P(User) = 0.005$
		\item Question: What is the probability that a randomly selected individual with a positive test is a drug user? $\implies P(User|+) = ?$
		\item 
				$P(User|+) = \frac{P(+|User)P(User)}{P(+)} = $
		\item $\frac{P(\text{+}\mid\text{User}) P(\text{User})}{P(\text{+}\mid\text{User}) P(\text{User}) + P(\text{+}\mid\overline{User}) P(\overline{User})} = \frac{0.99\times 0.005}{0.99\times 0.005 + 0.01\times0.995} \approx .332$ 
	
\end{itemize}
\end{frame}

\begin{frame}{Another example on Bayes rule}
\end{frame}


\begin{frame}{Bayes Rule for Machine Learning}
\begin{itemize}


    \item $P(A|B)P(B) = P(B|A)P(A)$
    \item Let us consider for a machine learning problem:
    \begin{itemize}
    	\item A = Parameters ($\theta$)
    	\item B = Data ($\mathcal{D}$)
    \end{itemize}
\item We can rewrite the Bayes rule as:
\begin{itemize}
	\item $P(\theta|\mathcal{D}) = \frac{P(\mathcal{D}|\theta)P(\theta)}{P(\mathcal{D})}$
	\item Posterior: 
	\item Prior:
	\item Likelihood
	\item 
\end{itemize}
\end{itemize}
\end{frame}

\begin{frame}{Likelihood}
\begin{itemize}
	\item Likelihood is a function of $\theta$
	\item Given a coin flip and 5 H and 1 T, what is more likely: P(H) = 0.5 or P(H) = 1
\end{itemize}
\end{frame}

\begin{frame}{Bayesian Learning is well suited for online settings}
content...
\end{frame}

\begin{frame}{Coin flipping}
\begin{itemize}
	\item Assume we do a coin flip multiple times and we get the following observation: \{H, H, H, H, H, H, T, T, T, T\}: 6 Heads and 4 Tails
	\item  What is $P(Head)$?
	\item Is your answer: 6/10. Why?
\end{itemize}

\end{frame}

\begin{frame}{Coin flipping: Maximum Likelihood Estimate (MLE)}
\begin{itemize}
	\item We have $\mathcal{D} = \{\data_1, \data_2, ...\data_{N}\}$ for $N$ observations where each $\mathcal{D}_i \in \{H, T\}$
	\item Assume we have $n_H$ heads and $n_T$ tails, $n_H + n_T = N$
	\item Let us have $P(H) = \theta, P(T) = 1-\theta$
	\item We have Likelihood, $L(\theta) = P(\mathcal{D}|\theta) = P(\data_1, \data_2, ..., \data_N|\theta)$
	\item Since observations are i.i.d., $L(\theta) = P(\data_1|\theta).P(\data_2|\theta) ... P(\data_N|\theta)$
\end{itemize}

\end{frame}


\begin{frame}{Coin flipping: Maximum Likelihood Estimate (MLE)}
\begin{itemize}
	\item  
\begin{align*}  
P(\data_i|\theta) =  \left
\{\begin{array}{lr} \theta, & \text{for~} \data_i =H \\
1-\theta, & \text{for~} \data_i = T
\end{array}\right.\
\end{align*}  
\item Thus, $L(\theta) = \theta^{n_H}\times (1-\theta)^{n_T}$
\item Log-Likelihood, $LL(\theta) = n_Hlog\theta + (n_T)(log(1-\theta))$
\item $\frac{\partial LL(\theta)}{\partial \theta} = \frac{n_H}{\theta} + \frac{n_T}{1-\theta}$
\item  For maxima, set derivative of LL to zero

\item 	$\frac{n_H}{\theta} + \frac{n_T}{1-\theta} = 0 $
\end{itemize}
\begin{tcolorbox}
	 $\theta = \frac{n_H}{n_H + n_T}$
\end{tcolorbox}

\end{frame}

\begin{frame}{Maximum A Posteriori estimate (MAP)}
\begin{itemize}


\item \textbf{MLE does not handle prior knowledge}: What if we know that our coin is biased towards head?
\item \textbf{MLE can overfit}: What is the probability of heads when we have observed 6 heads and 0 tails?
\end{itemize}

\end{frame}


\begin{frame}{Maximum A Posteriori estimate (MAP)}
Goal: Maximize the Posterior
\begin{tcolorbox}
$\hat{\theta}_{MAP} = \underset{\theta}{\mathrm{argmin~}}
P(\theta|\data)$\\
$\hat{\theta}_{MAP}= \underset{\theta}{\mathrm{argmin~}}
P(\data|\theta)P(\theta)$
\end{tcolorbox}

\end{frame}

\begin{frame}{Prior distributions}
\end{frame}

\begin{frame}{Beta Distribution}
\end{frame}

\begin{frame}{Beta Distribution}
\end{frame}

\begin{frame}{Coin toss: MAP estimate}
\end{frame}

\begin{frame}{Linear Regression: MLE}
\end{frame}

\end{document}